\chapter{Data Representation and Storage\label{sec:data}}
\setcounter{examples}{1}

Computers use binary numbers because they are made up of electronic components called transistors, 
which can be either in an on or off state. By using a \gls{base-2}, which only 
has two digits (0 and 1), computers can represent and process information using these on and 
off states of transistors.

The \gls{base-2} numeral system is a positional notation with a base of 2. Each digit is referred to as 
a \gls{bit}, or binary digit. Because of its straightforward implementation in digital electronic 
circuitry using logic gates, the binary system is used by almost all modern computers and 
computer-based devices, as a preferred system of use, over various other human techniques of 
communication, because of the underlying transitor-level components.

Numbers represented in computers are constrained by a finite amount of space. Your hard disk drive 
has a finite number of bits that can be stored on it. For example, a 256 gigabyte (GB) hard disk drive 
can store 96,000,000,000 bits. This is a lot of data. However, there are an infinite number of natural
numbers, and it is possible to define a number that would be too large to store on the hard drive such as 
$2^{96000000000}$.

The microprocessor uses a bus to exchange information. This bus has a finite capacity. So, even if 
there were a hard drive with the ability to store all natural numbers (infinite capacity), you can 
only retrieve a finite number of bits from the drive per second. For example, the \gls{x64} \gls{isa} 
can retrieve 64-bits (though it can do this at a rate of millions or billions of times per second).

Finally, a hard drive or memory does not store a single number. For it to be useful we will need it 
to store a set of numbers. So, there must be 
some scheme for organizing the data into useable chunks. These chunks will need to be organized or 
indexed in a way that the microprocessor can reach it through the bus. This is called an \gls{address}. 
This leads to the following questions:
%
\begin{enumerate}
    \item If you have a finite number of bits, what happens when you have a number that is too large to be properly stored?
    \item How can you organize data into smaller chunks in memory so that they are useable?
    \item How can you index the chunks of memory so that they can be referenced by the microprocessor?
    \item Considering that the bus is finite, when chunks of memory are retrieved, in what order are they passed?
\end{enumerate}
%
These questions will be the focus of this chapter. 


\section{Integer Overflow\label{sec:data:overflow}}
With digital computers, any chunk of data will have a finite number of bits. It is important to note the largest positive 
integer that can be stored for a binary sequence of a given length to avoid a concept called \gls{overflow}. 
For \gls{unsigned} numbers this is defined as: %

%
\begin{equation} \label{eq:sizeofunsignedint}
    2^n - 1
\end{equation}
%
Where $n$ is the number of digits in the binary sequence.

In computer programming, an integer overflow occurs when an arithmetic operation attempts to create a numeric value that is 
outside of the range that can be represented with a given number of digits. The most common result of an overflow is that the 
least significant representable digits of the result are stored; the result is said to wrap around the maximum (i.e. modulo a 
power of the radix, usually two in modern computers, but sometimes ten or another radix).

An overflow condition may give results leading to unintended behavior. In particular, if the possibility has not been anticipated, 
overflow can compromise a program's reliability and security. For some applications, such as timers and clocks, wrapping on 
overflow can be desirable. On some processors like graphics processing units (GPUs) and digital signal processors (DSPs) which 
support saturation arithmetic, overflowed results would be "clamped", i.e. set to the minimum or the maximum value in the 
representable range, rather than wrapped around. 

Microprocessors have a special status bit called the carry flag. It is set when the result of an addition or subtraction, 
considering the operands and result as unsigned numbers, does not fit in the given number of bits. This indicates an overflow 
with a carry or borrow from the most significant bit. An immediately following add with carry or subtract with borrow operation 
would use the contents of this flag to modify a register or a memory location that contains the higher part of a multi-word value. 
In later sections we will discuss how this is used by the microprocessor to compare two values.


\begin{figure}[h]
    \example{What is the largest number that you can store in an unsigned C-language \texttt{int} data type? %
    Note that \texttt{int} is 32 bits, so $n=32$
    \begin{align*}%
        & 2^n-1 \\
        & 2^{32} - 1 = 4,294,967,295
    \end{align*}%
    Perhaps this is somewhat shocking. A C-language \texttt{int} cannot hold numbers greater than four billion.
    }
\end{figure}

\begin{figure}[h]
    \example{Suppose that you want to use binary numbers to encode specific letters of the Spanish language %
    alphabet which has 27 letters. At least how many bits will you need to uniquely encode each letter?
    \begin{align*}
        & 2^n-1 \geq 27 \\
        & 2^n \geq 28 \\
        & n \geq \log_2 28 \\
        & n \geq 4.8 \approx 5
    \end{align*}
    You cannot have a fractional number of digits, so it must be at least 5 bits.
    }
\end{figure}

\section{Bit Organization\label{sec:data:bits}}

Long sequences of bits can be difficult to understand. One method for aggregating \gls{base-2} was covered (hexadecimal). 
Another method groups numbers into units, similarly to measurement units. For example, in English units one yard is 
three feet, and one foot is twelve inches. There is a similar system for organizing bits. The most common method is 
a byte. A byte is eight bits and abbreviated with a capital B. Bytes are abbreviated with decimal metric prefixes. For 
example, 1 KB is 1,000 bytes or 8,000 bits.

It is sometimes necessary to specify a chunk of data greater than a byte. This chunk is formally called a word. The 
length of a word varies from system to system. For example, the PlayStation 2's MIPS microprocessor had a word length of 
32 bits or 4 bytes. Modern \gls{x64} systems use a word length of 2 bytes, due to being backward compatible with systems 
from the 1970's--when 2 bytes was an appropriately sized chunk of data. The word length is arbitrary and does not necessarily
indicate the performance of a system.