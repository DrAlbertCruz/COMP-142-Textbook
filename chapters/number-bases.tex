
\chapter{Number Bases\label{sec:numbers}}
\setcounter{examples}{1}

A positional system is a numeral system in which the contribution of a digit to the value of a number 
is the value of the digit multiplied by a factor determined by the position of the digit. In early 
numeral systems, such as Roman numerals, a digit has only one value: I means one, X means ten and C 
a hundred (however, the value may be negated if placed before another digit). In modern positional 
systems, such as the decimal system, the position of the digit means that its value must be multiplied 
by some value: in 555, the three identical symbols represent five hundreds, five tens, and five units, 
respectively, due to their different positions in the digit string. 

Among the earliest systems was the Babylonian numeral system. It used base 60. It was the first 
positional system to be developed, and its influence is present today in the way time and angles are 
counted in tallies related to 60, such as 60 minutes in an hour and 360 degrees in a circle. Today, 
the Hindu-Arabic numeral system (\gls{base-10}) is the most commonly used system globally. However,  
the binary numeral system (\gls{base-2}) is used in almost all computers and electronic devices because it 
is easier to implement efficiently in electronic circuits. 

\begin{figure}[b]\centering
    \begin{tabular}{rcccccccccc}
        Western Arabic & 0 & 1 & 2 & 3 & 4 & 5 & 6 & 7 & 8 & 9 \\
        Eastern Arabic & \textarab{0} & \textarab{1} & \textarab{2} & \textarab{3} & \textarab{4} & \textarab{5} & \textarab{6} & \textarab{7} & \textarab{8} & \textarab{9} \\
        East Asian & \CJKnumber{0} & \CJKnumber{1} & \CJKnumber{2} & \CJKnumber{3} & \CJKnumber{4} & \CJKnumber{5} & \CJKnumber{6} & \CJKnumber{7} & \CJKnumber{8} & \CJKnumber{9} \\
    \end{tabular}
    \caption{Many cultures use \gls{base-10} for arithmetic, differing only by the symbol used to represent each digit. Numerals used in Western culture are 
    formally called Western Arabic numerals\label{fig:base10}}
\end{figure}

\section{Base-2\label{sec:data:bases}}

You can think of a number base as the way to represent a number. The value of the number does not 
change when transitioning between number bases. Most people use a \gls{base-10}, also known as decimal.
While symbols may change for each digit, the intrinsic value of a number remains the same (see Figure \ref{fig:base10}). 
For example, 667 is a decimal number. When you work problems that have numbers of different bases, 
write the number in subscript to avoid confusion, for example: $667_{10}$ the number 667 written in 
decimal. Binary numbers are \gls{base-2}. $667_{10}$ in \gls{base-2} is representated as $1010011011_2$. 

When spoken, binary numerals are usually read digit-by-digit, in order to distinguish them from 
decimal numerals. For example, $1010011011_2$ is pronounced \textit{one zero one zero zero one 
one zero one one}. It would be confusing to refer to the number as \textit{one billion ten million 
eleven thousand and eleven} which represents a different value. Note that the number base should 
not change the value of a number.

The order of the digits in a binary number represents their power of two. The right-most digit is 
called the \gls{lsb} represents the power of $2^0$. The left-most digit is called 
the \gls{msb} represents the power of $2^{n-1}$ where $n$ is the length of the 
sequence. For example, with $1010011011_2$ the LSB is 1, representing $2^0$, and the MSB is $2^9$ 
because the sequence is 10 bits long. 

Counting from $n-1$ to zero, if you have a sequence of numbers $m = a_{n-1}a_{n-2}a_{n-3}...a_2a_1a_0$ 
where $a_{n-1}$ is the \gls{msb} and $a_0$ is the \gls{lsb}, you can obtain the value of this number 
by summing up powers of two:

\begin{equation}\label{eq:base2}
    m = \sum_{i=0}^{n-1} a_i \times 2^i
\end{equation}

\begin{figure}[h!]
    \example{Convert $1010011011_2$ to decimal by expanding the powers of two. The most straightforeward %
    way to convert this number is to expand it in of powers following Equation \ref{eq:base2}:%
    %
    \begin{equation}\label{sec:data:bases:667bin}%
        1 \times 2^9 + 0 \times 2^8 + 1 \times 2^7 + 0 \times 2^6 + 0 \times 2^5 + 1 \times 2^4 
        + 1 \times 2^3 + 0 \times 2^2 + 1 \times 2^1 + 1 \times 2^0
    \end{equation}%
    Note that we write out all powers of two starting with $n-1$. In this case $n=10$ because there %
    ten digits. Begining with the \acrshort{msb} and ending with the \acrshort{lsb} copy 0 or 1 %
    based on the corresponding digit in the binary representation of the number. This may seem cumbersome %
    but after some time you will memorize the powers of 2 and it will become easier.}
\end{figure}

Expanding the number in terms of powers of two is the simplest way to convert a decimal number %
to binary (such as in Equation \ref{sec:data:bases:667bin}). An easier way is to prepare a table with the %
power of two that is just less than the magnitude of the number you are converting. For $667_{10}$, this is %
$512_{10}$ or $2^9$:

\vspace{1em} %
\begin{tabular}{|l|l|l|l|l|l|l|l|l|l|l|}\hline
Power & $2^9$ & $2^8$ & $2^7$ & $2^6$ & $2^5$ & $2^4$ & $2^3$ & $2^2$ & $2^1$ & $2^0$ \\\hline\hline
Decimal & 512 & 256 & 128 & 64 & 32 & 16 & 8 & 4 & 2 & 1 \\\hline
Digit &  &  &  &  &  &  &  &  &  &  \\\hline
\end{tabular}
\vspace{1em}

Start from left to right. If you can subtract the number without it becoming negative, indicate 1 for the digit. %
Then, carry out the subtraction and use this new value for the next column. If you cannot carry out the subtraction % 
without the number becoming negative, skip to the next column. Repeat this procedure for the next column. %
$667-512=155$, so we can indeed subtract $512_2$ from $667_2$. We can note 1 in the digit, and use the value of $155$ %
for the next column. $155-256<0$. We cannot subtract without the number becoming negative, so we note 0 as the digit %
for this current column. However, $155-128=27$. So we can note 1 in the digit for this next column. And so on until %
the last digit. If you have any remaining value beyond the right-most column you have made a mistake, check your work. %

\vspace{1em} %
\begin{tabular}{|l|l|l|l|l|l|l|l|l|l|l|}\hline
Power & $2^9$ & $2^8$ & $2^7$ & $2^6$ & $2^5$ & $2^4$ & $2^3$ & $2^2$ & $2^1$ & $2^0$ \\\hline\hline
Decimal & 512 & 256 & 128 & 64 & 32 & 16 & 8 & 4 & 2 & 1 \\\hline
Digit & 1 & 0 & 1 & 0 & 0 & 1 & 1 & 0 & 1 & 1 \\\hline
\end{tabular}
\vspace{1em}

\subsection{Divide-by-2\label{sec:numbers:divideby2}}

There is a more formal way to carry out this procedure, called the \textbf{divide-by-2} method. Essentially, with the previous 
algorithm the digit column is noting if there would be a remainder when carrying out integer division by its respective power of 2. 
An alternative way to convert the number is as follows.

Start with the number you want to convert. Perform an integer division by 2. Note that an integer division does not produce a fractional
 result. It should produce an integer and a remainder if the divisor cannot cleanly divide the dividend. Write the remainder to the 
 right of the calculation, and write the result \textit{below} your calculation. Continue dividing by 2 until you reach a dividend of 1. 
 The binary representation of the number is read from top-to-bottom of the remainder values.

\section{Generalized Number Bases\label{sec:numbers:bases}}

\Gls{base-10} and \gls{base-10} are not the only number bases. The base of a number can be any 
integer. Some cultures measure units in dozens, or base-12. Time is measured in base-12 or 
base-24 depending on which country you are in. Other common bases used in computing are \gls{base-8} and 
\gls{base16}. \Gls{base-8} is often used in Unix system permissions to represent the permissions of
different user groups. \Gls{base-16} is often used to compactly represent binary numbers to 
reduce human error when reading, understanding or copying data.

In Unix system permissions on Unix-like file systems are defined in the POSIX.1-2017 standard,
which uses three classes known as user, group, and others. Further, each class of user can attempt 
to read, write and execute a specific file. It is able to or it is not (it is binary), so  
\vfill\clearpage
\section*{Homework Questions}

\small
\begin{multicols*}{2}
   \begin{enumerate}[label=\thechapter.\arabic*]
    \item Conduct a search on the historical basis for binary numbers before they were used in %
    computing and explain how they were used by society.
    \item Explain in your own words the following concepts:
    \begin{enumerate}
        \item Most significant bit
        \item Least significant bit
        \item The largest unsigned number that can be stored in $n$ bits.
    \end{enumerate}
    \item Define the divide-by-2 method in pseudo-code.
    \item What is the largest unsigned number that can be stored in the following data types?
    \begin{enumerate}
        \item \texttt{char}
        \item 2-byte \texttt{int}
        \item 48-bit integer
    \end{enumerate}
    \item Convert the following decimal numbers to binary:
    \begin{multicols*}{2}
        \begin{enumerate}
            \item 0
            \item 1
            \item 23
            \item 59
            \item 100
            \item 1200
            \item 1092
            \item 1000000
        \end{enumerate}
    \end{multicols*}
    \item Convert the following binary numbers to decimal using the table method:\label{hw:tablemethod}
    \begin{enumerate}
        \item 0
        \item 1
        \item 1010
        \item 1111
        \item 1110101
        \item 1010100101
        \item 1000000000
        \item 1000100101
    \end{enumerate}
    \item Repeat Question \ref{hw:tablemethod} using the divide-by-2 method. 
   \end{enumerate} 
\end{multicols*}
