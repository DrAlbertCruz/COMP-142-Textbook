
\chapter{Number Bases\label{sec:numbers}}
\setcounter{examples}{1}

A positional system is a numeral system in which the contribution of a digit to the value of a number 
is the value of the digit multiplied by a factor determined by the position of the digit. In early 
numeral systems, such as Roman numerals, a digit has only one value: I means one, X means ten and C 
a hundred (however, the value may be negated if placed before another digit). In modern positional 
systems, such as the decimal system, the position of the digit means that its value must be multiplied 
by some value: in 555, the three identical symbols represent five hundreds, five tens, and five units, 
respectively, due to their different positions in the digit string. 

Among the earliest systems was the Babylonian numeral system. It used base 60. It was the first 
positional system to be developed, and its influence is present today in the way time and angles are 
counted in tallies related to 60, such as 60 minutes in an hour and 360 degrees in a circle. Today, 
the Hindu-Arabic numeral system (\gls{base-10}) is the most commonly used system globally. However,  
the binary numeral system (\gls{base-2}) is used in almost all computers and electronic devices because it 
is easier to implement efficiently in electronic circuits. 

The simplest numeral system is the unary numeral system, in which every natural number is represented by 
a corresponding number of symbols. \Gls{tally marks} represent one such system still in common use. In Western 
culture it is common to cross four tally marks every multiple of five. However, a cross still uses a line 
and technically meets the requirements for a unary system. The unary system is only useful for small numbers, 
and finds use in one data compression algorithm (Elias gamma encoding).

More elegant is a positional system, also known as place-value notation. For example, in \gls{base-10}, ten 
different digits 0, ..., 9 are used and the position of a digit is used to signify the power of ten that the 
digit is to be multiplied with, as in $304 = 3 \times 100 + 0 \times 10 + 4 \times 1$. Zero, which is not 
needed in the other systems, is of crucial importance here, in order to be able to skip a power.

Indian mathematicians are credited with developing \gls{base-10}.
\gls{aryabhata} developed the place-value notation in the 5th century CE and a century later 
\gls{brahmagupta} introduced the symbol for zero. The system slowly spread to other surrounding regions 
like Arabia due to their commercial and military activities with India. Middle-Eastern mathematicians 
extended the system to include negative powers of 10 (fractions), as recorded in a treatise by Syrian 
mathematician \gls{abulhasan} in 952-953 CE, and the decimal point notation was introduced by 
\gls{sindibnali}, who also wrote the earliest treatise on Arabic numerals. \Gls{base-10} then 
spread to Europe due to merchants trading, and the digits used in Europe are called Western Arabic numerals, 
as they learned them from the Arabs--though it was invented in India. 


\begin{figure}[t]\centering
    \begin{tabular}{rcccccccccc}
        Western Arabic & 0 & 1 & 2 & 3 & 4 & 5 & 6 & 7 & 8 & 9 \\
        Eastern Arabic & \textarab{0} & \textarab{1} & \textarab{2} & \textarab{3} & \textarab{4} & \textarab{5} & \textarab{6} & \textarab{7} & \textarab{8} & \textarab{9} \\
        East Asian & \CJKnumber{0} & \CJKnumber{1} & \CJKnumber{2} & \CJKnumber{3} & \CJKnumber{4} & \CJKnumber{5} & \CJKnumber{6} & \CJKnumber{7} & \CJKnumber{8} & \CJKnumber{9} \\
    \end{tabular}
    \caption{Many cultures use \gls{base-10} for arithmetic, differing only by the symbol used to represent each digit. Numerals used in Western culture are 
    formally called Western Arabic numerals\label{fig:base10}}
\end{figure}

\section{Base-2\label{sec:data:bases}}

The modern binary number system was studied in Europe in the 16th and 17th centuries by Thomas 
Harriot, Juan Caramuel y Lobkowitz, and Gottfried Leibniz. However, systems related to binary 
numbers have appeared earlier in multiple cultures including ancient Egypt, China, and India. 
Leibniz was specifically inspired by the \gls{iching}, a classical Chinese text from the 9th century BCE. 

You can think of a number base as the way to represent a number. The value of the number does not 
change when transitioning between number bases. Most people use a \gls{base-10}, also known as decimal.
While symbols may change for each digit, the intrinsic value of a number remains the same (see Figure \ref{fig:base10}). 
For example, 667 is a decimal number. When you work problems that have numbers of different bases, 
write the number in subscript to avoid confusion. For example $667_{10}$ is the number 667 written in 
decimal. Binary numbers are \gls{base-2}. $667_{10}$ in \gls{base-2} is representated as $1010011011_2$. 

When spoken, binary numerals are usually read digit-by-digit, in order to distinguish them from 
decimal numerals. For example, $1010011011_2$ is pronounced \textit{one zero one zero zero one 
one zero one one}. It would be confusing to refer to the number as \textit{one billion ten million 
eleven thousand and eleven} which represents a different value. Note that the number base should 
not change the value of a number.

The order of the digits in a binary number represents their power of two. The right-most digit is 
called the \gls{lsb} represents the power of $2^0$. The left-most digit is called 
the \gls{msb} represents the power of $2^{n-1}$ where $n$ is the length of the 
sequence. For example, with $1010011011_2$ the LSB is 1, representing $2^0$, and the MSB is $2^9$ 
because the sequence is 10 bits long. Generally, when performing operations on binary numbers the 
\gls{lsb} of both operands must be aligned. By convention, it is on the right side.

Counting from $n-1$ to zero, if you have a sequence of numbers $m = a_{n-1}a_{n-2}a_{n-3}...a_2a_1a_0$ 
where $a_{n-1}$ is the \gls{msb} and $a_0$ is the \gls{lsb}, you can obtain the value of this number 
by summing up powers of two:

\begin{equation}\label{eq:base2}
    m = \sum_{i=0}^{n-1} a_i \times 2^i
\end{equation}

\begin{figure}[t]
    \example{Convert $1010011011_2$ to decimal by expanding the powers of two. The most straightforeward %
    way to convert this number is to expand it in of powers following Equation \ref{eq:base2}:%
    %
    \begin{equation}\label{sec:data:bases:667bin}%
        1 \times 2^9 + 0 \times 2^8 + 1 \times 2^7 + 0 \times 2^6 + 0 \times 2^5 + 1 \times 2^4 
        + 1 \times 2^3 + 0 \times 2^2 + 1 \times 2^1 + 1 \times 2^0
    \end{equation}%
    Note that we write out all powers of two starting with $n-1$. In this case $n=10$ because there %
    ten digits. Begining with the \acrshort{msb} and ending with the \acrshort{lsb} copy 0 or 1 %
    based on the corresponding digit in the binary representation of the number. This may seem cumbersome %
    but after some time you will memorize the powers of 2 and it will become easier.}
\end{figure}

Expanding the number in terms of powers of two is the simplest way to convert a decimal number %
to binary (such as in Equation \ref{sec:data:bases:667bin}). An easier way is to prepare a table with the %
power of two that is just less than the magnitude of the number you are converting. For $667_{10}$, this is %
$512_{10}$ or $2^9$:

\vspace{1em} %
\begin{tabular}{|l|l|l|l|l|l|l|l|l|l|l|}\hline
Power & $2^9$ & $2^8$ & $2^7$ & $2^6$ & $2^5$ & $2^4$ & $2^3$ & $2^2$ & $2^1$ & $2^0$ \\\hline\hline
Decimal & 512 & 256 & 128 & 64 & 32 & 16 & 8 & 4 & 2 & 1 \\\hline
Digit &  &  &  &  &  &  &  &  &  &  \\\hline
\end{tabular}
\vspace{1em}

Start from left to right. If you can subtract the number without it becoming negative, indicate 1 for the digit. %
Then, carry out the subtraction and use this new value for the next column. If you cannot carry out the subtraction % 
without the number becoming negative, skip to the next column. Repeat this procedure for the next column. %
$667-512=155$, so we can indeed subtract $512_2$ from $667_2$. We can note 1 in the digit, and use the value of $155$ %
for the next column. $155-256<0$. We cannot subtract without the number becoming negative, so we note 0 as the digit %
for this current column. However, $155-128=27$. So we can note 1 in the digit for this next column. And so on until %
the last digit. If you have any remaining value beyond the right-most column you have made a mistake, check your work. %

\vspace{1em} %
\begin{tabular}{|l|l|l|l|l|l|l|l|l|l|l|}\hline
Power & $2^9$ & $2^8$ & $2^7$ & $2^6$ & $2^5$ & $2^4$ & $2^3$ & $2^2$ & $2^1$ & $2^0$ \\\hline\hline
Decimal & 512 & 256 & 128 & 64 & 32 & 16 & 8 & 4 & 2 & 1 \\\hline
Digit & 1 & 0 & 1 & 0 & 0 & 1 & 1 & 0 & 1 & 1 \\\hline
\end{tabular}
\vspace{1em}

\subsection{Divide-by-2\label{sec:numbers:divideby2}}

There is a more formal way to carry out this procedure, called the \gls{divide-by-2} method. Essentially, with the previous 
algorithm the digit column notes if there would be a remainder when carrying out integer division by its respective power of 2. 
Another way to convert the number is as follows.

Start with the number. Perform an integer division by 2. An integer division does not produce a fractional result. It should 
produce an integer and a remainder if the divisor cannot cleanly divide the dividend. Write the remainder to the right of the 
calculation, and write the result as the dividend of your next calculation. Continue until the result is zero. The binary 
representation of the number is read from top-to-bottom of 
the remainder values.

\begin{align*}
	667 \div 2 = 333  	&& R & 1 \\
	333 \div 2 = 166 	&& R & 1 \\
	166 \div 2 = 83 	&& R & 0 \\
	83  \div 2 = 41 	&& R & 1 \\
	41	\div 41 = 20 	&& R & 1 \\
	20 \div 2 = 10 		&& R & 0 \\
	10 \div 2 = 5 		&& R & 0 \\
	5 \div 2 = 2 		&& R & 1 \\
	2 \div 2 = 1 		&& R & 0 \\
	1 \div 2 = 0		&& R & 1
\end{align*}

$667 \div 2$ has a remainder of 1 because it is odd, so write this to the right of the division (R 1), and write the result, 
$333$ below as the dividend of the new operation. Repeat this step. $333 \div 2$ also has a remainder of 1, and its result is 
166. $166$ is even and has a remainder of 0. And so on until we reach $2 \div 1$.To obtain the answer read from the top down. 
The bottom digit is the LSB and the top digit is the MSB. The result is $1101100101_2$ which is consistent with our previous 
examples.

\begin{figure}[t]
    \example{Convert $1_{10}$ to binary. % 
    \begin{align*}
        1 \div 2 = 0 && R & 1 
    \end{align*}
    Note that this is a base case of the algorithm. A result of zero means stop. The value 1 is the same in binary as it is %
    in decimal.}
\end{figure}

\begin{figure}[t]
    \example{Convert $1025_{10}$ to binary. % 
    \begin{align*}
        1025 \div 2 = 512 && R & 1 \\
        512 \div 2 = 256 && R & 0 \\
        256 \div 2 = 128 && R & 0 \\
        128 \div 2 = 64 && R & 0 \\
        64 \div 2 = 32 && R & 0 \\
        32 \div 2 = 16 && R & 0 \\
        32 \div 2 = 16 && R & 0 \\
        16 \div 2 = 8 && R & 0 \\
        8 \div 2 = 4 && R & 0 \\
        4 \div 2 = 2 && R & 0 \\
        2 \div 2 = 1 && R & 0 \\
        1 \div 2 = 0 && R & 1 
    \end{align*}
    $1025_{10}$ is equal to $100000000001_2$. Notice that $1\div2$ is consistently the termination condition with \gls{base-2}. It is a common mistake to stop prematurely at $2\div2$.}
\end{figure}

\section{Generalized Number Bases\label{sec:numbers:bases}}

\Gls{base-10} and \gls{base-2} are not the only number bases. The base of a number can be any 
integer. Some cultures measure units in dozens, or base-12. Hours are measured in base-12 or 
base-24 depending on which country you are in; minutes, base-60. Other common bases used in 
computing are \gls{base-8} and \gls{base-16}. \Gls{base-8} is used in Unix system permissions 
to represent the permissions of different user classes. \Gls{base-16} is used to compactly 
represent binary numbers to reduce human error when reading, understanding or copying data.

The equation defining expansions of \gls{base-2} can be generalized to any base $b$. Counting from $n-1$ 
to zero, if you have a sequence of numbers $m = a_{n-1}a_{n-2}a_{n-3}...a_2a_1a_0$ 
where $a_{n-1}$ is the \gls{msb} and $a_0$ is the \gls{lsb}, you can obtain the value of this number 
by summing up powers of $b$:
%
\begin{equation}\label{eq:basen}
    m = \sum_{i=0}^{n-1} a_i \times b^i
\end{equation}
%
Furthermore the divide-by-2 algorithm can be generalized as a divide-by-$b$ algorithm if you divide by the base $b$ rather than 2.

\begin{figure}[t]
    \example{Convert $1234_{10}$ to base-3. % 
    \begin{align*}
        1234 \div 3 = 411 && R & 1 \\
        411 \div 3 = 137 && R & 0 \\
        137 \div 3 = 45 && R & 2 \\
        45 \div 3 = 15 && R & 0 \\
        15 \div 3 = 5 && R & 0 \\
        5 \div 3 = 1 && R & 2 \\
        1 \div 3 = 0 && R & 1
    \end{align*}
    $1234_{10}$ is equal to $1200201_3$. Base-3 is not a commonly used base and this is a contrived exampled to demonstrate that 
    the \gls{divide-by-2} algorithm works for any arbitrary number base.}
\end{figure}

\begin{figure}[t]
    \example{How many donuts are in a box of twelve dozen donuts? Assuming that one means
    that there are $12_{12}$ donuts and not being redundant or $12 \times 12$ donuts. %
    \begin{align*} %
        m & = \sum_{i=0}^{n-1} a_i \times 12^i \\
        m & = 1 \times 12^1 + 2 \times 12^0  \\
        m & = 12 + 2 =  14 \\
    \end{align*}%
    There are $14_{10}$ donuts in a box of $12_{12}$ donuts. 
    }
\end{figure}

\begin{figure}[t]
    \example{Convert the tally marks llll to decimal using Equation \ref{eq:basen}. %
    Unary numbering systems can be thought of as base-1.
    \begin{align*} %
        m & = \sum_{i=0}^{n-1} a_i \times 1^i \\
        m & = 1 \times 1^3 + 1 \times 1^2 + 1 \times 1^1 + 1 \times 1^0  \\
        m & = 1 + 1 + 1 + 1 = 4 \\
    \end{align*}%
    1 taken to any power remains 1. 
    }
\end{figure}

\subsection{Octets}

File permissions on Unix-like file systems are defined in the POSIX.1-2017 standard. There 
are three types of users, called classes: user, group, and others. Further, each class of user can 
read ($r$), write ($w$) or execute a file ($x$). Each permission (read, write, execute) is binary. 
You can or cannot perform the action. If you order them ($rwx$) as a bit string the number can 
encode the specific combination of allowed/disallowed priveledges. Continuing with the example, 
$r\neg w \neg x = 100_2 = 4_{8}$. Allowing all priveledges can be encoded 
as $111_2 = 7_{8}$.  Allowing none would be the deassertion of all bits or $0$.

There are a maximum of 8 combinations (remember to include 0), thus it is a 
\gls{base-8} system. In practice it is called an octet. This system is not used for arithmetic. Rather,
the number is used as a category to indicate the permissions for a class. Each binary digit encodes 
the permissions in the order given above ($rwx$). Because there are three classes, 
Unit-like permissions are given with three octets. For example, $777_8$ allows all permissions 
for all user classes.

\begin{figure}[t]
    \example{Serving files for the Internet requires special permissions. The user should be allowed read %
    and write files, but group and others should only be allowed to read files. For folders, the user should %
    be allowed all permissions. But, group and others should only be allowed to execute the folder. Give %
    the octets for the files and folders for serving files to the Internet. \\\\%
    %
    The order of classes is user, group and others. The user can read and write files, but not execute them. %
    This is defined as $110_2 = 6_8$. Group and others can only read the files but not write or execute them. %
    This is defined as $010_2 = 4_8$. Thus, the octet we need for files on a web server is $644_8$. \\\\%
    %
    For folders, the user has all priveledges. We know this is $7_8$. However, group and others can only %
    execute the folder. This is defined as $001_2 = 1_8$. The octet we need for folders on the web server is %
    $711_8$.
    }
\end{figure}

\begin{figure}[t]
    \example{In binary, the octet of permissions for a Unix file is $111111100_2$. What can each group %
    do with this file? The first octet or three bits are the user permissions. $111_2 = 7_8$, so the user %
    who owns the file can do anything with it--similarly for the group. However, the last three bits are %
    $100_2 = 4_8$. So others can only read the file. %
    }
\end{figure}

\subsection{Hexadecimal\label{sec:numbers:bases:hex}}


The hexadecimal (also \gls{base-16} or simply hex) numeral system is a positional numeral system that 
represents numbers using a base of 16. Unlike the decimal system representing numbers using 10 symbols, 
hexadecimal uses 16 distinct symbols, most often the symbols 0-9 to represent values 0-9, and A-F to 
represent values from 10 to 15. 

\Gls{base-16} was used in the early 20th century as traditional units of weight in China. %
Currently, software developers and system designers widely use hexadecimal numbers because they provide a human-
friendly representation of binary-coded values. Each hexadecimal digit represents four bits (binary digits), 
also known as a \gls{nibble}. For example, a 4-bit nibble can have values ranging from 0000 to 1111 in 
binary form, which can be conveniently represented as 0 to F in hexadecimal. 


It is possible for you to convert from decimal to \gls{base-16} using the \gls{divide-by-2} algorithm. 
As a developer you will not often be converting decimal to hexadecimal. Instead, you will use 
hexadecimal to abbreviate long texts of binary. There is no need to convert from binary to decimal 
then from decimal to hexadcimal using \gls{base-16}. Due to some unique property of the positional
numbering system, binary numbers can be converted to hexadecimal by table lookup (see Table \ref{tab:hex}). 

\begin{table}[b]\centering
    \caption{Table for converting between binary and hexadecimal.\label{tab:hex}}
    \begin{tabular}{|l|l|}\hline 
        Binary & Hexadecimal \\\hline\hline  
        0000 & 0 \\\hline 
        0001 & 1\\\hline 
        0010 & 2\\\hline 
        0011 & 3\\\hline 
        0100 & 4\\\hline 
        0101 & 5\\\hline 
        0110 & 6\\\hline 
        0111 & 7\\\hline 
        1000 & 8\\\hline 
        1001 & 9\\\hline 
        1010 & A\\\hline 
        1011 & B \\\hline 
        1100 & C\\\hline 
        1101 & D\\\hline 
        1110 & E \\\hline 
        1111 & F\\\hline 
    \end{tabular}
\end{table}

In mathematical contexts, a subscript is used to specify the base. For example, the decimal value $54,076_{10}$
would be expressed in hexadecimal as $D33C16_{16}$. In programming, a number of notations are used to denote
hexadecimal numbers, usually involving a prefix. The prefix \texttt{0x} is used in C language, which would 
denote this value as \texttt{0xD33C}. 

\section{Arithmetic in Varying Number Bases\label{sec:numbers:binarith}}

Carry arithmetic in decimal extends to any positional numbering system, including . Recall that 
in carry arithmetic, the \gls{lsb} is aligned on the right, and you add single digit numbers 
digit by digit. If the result is greater than one digit, the second digit is carried as a third 
operand for the subsequent single-digit-addition operation. For example, $123_{10} + 459_{10}$:

\vspace{1em}
\begin{tabular}{llll}
      &   & 1 &   \\
      & 1 & 2 & 3 \\
    + & 4 & 5 & 9 \\\hline
      & 5 & 8 & 2
\end{tabular}
\vspace{1em}

With the \gls{lsb}, $3+9=12$, and the $1$ is carried into the $2+5$ operation. This same algorithm 
can be used in binary. For example, $101_2 + 111_2$:

\vspace{1em}
\begin{tabular}{llll}
      & 1  & 1  &   \\
      & 1 & 0 & 1 \\
    + & 1 & 1 & 1 \\\hline
     1 & 1 & 0 & 0
\end{tabular}
\vspace{1em}

Note that in binary $1_2 + 1_2 = 10_2$, resulting in a carry-out. A similar carry out occurs with, 
$1_2 + 1_2 + 1_2 = 11_2$. You can even apply this logic to hexadecimal numbers. Consider 
$1AD_{16} + 20_{16}$. 

\vspace{1em}
\begin{tabular}{llll}
      &   &   &   \\
      & 1 & A & D \\
    + &   & 2 & 0 \\\hline
      & 1 & C & D
\end{tabular}
\vspace{1em}

This particular example did not involve any carry outs. When adding $A_{16} + 2_{16}$ it may 
be helpful to think of the numbers in decimal to determine which hexadecimal symbol results 
and if a carry out needs to occur (if the result is greater than 16). $A_{16}$ corresponds to 
$10_{10}$ so $10_{10}+2_{10}=12_{10}$. The symbol for 12 in hexadecimal is the letter C. An even 
quicker way is to note that an addition of two should increment the digit by two, and C 
is two letters after A. 

\vfill\clearpage
\section*{Homework Questions}

\small
\begin{multicols*}{2}
   \begin{enumerate}[label=\thechapter.\arabic*]
    \item Conduct a literature search on the following individuals and describe their contributions to decimal and/or binary numbers.
        \begin{enumerate}
            \item \gls{aryabhata}
            \item \gls{brahmagupta}
            \item \gls{abulhasan}
            \item \gls{sindibnali}
            \item Thomas Harriot
            \item Gottfried Leibniz
        \end{enumerate}
    \item Conduct a literature search on the historical basis for binary numbers before they were used in %
    computing and explain how they were used by society.
    \item Conduct a literature search on \gls{iching} and draw parallels to its concepts and \gls{base-2}.
    \item Conduct a literature search on why hexadecimal and binary number conversions can be done using a table while other common number bases cannot. 
    \item Conduct a literature search on the historical basis of base-60. How does it compare to a modern base-10 system?
    \item Explain in your own words the following concepts:
    \begin{enumerate}
        \item Most significant bit
        \item Least significant bit
        \item Positional notation 
        \item Zero 
        \item Tally marks
    \end{enumerate}
    \item Calculate the first 20 powers of 2.
    \item Write out the binary to hexadecimal conversion table. 
    \item Convert the following decimal numbers to binary using a table.
    \begin{multicols*}{2}
        \begin{enumerate}
            \item 0
            \item 1
            \item 23
            \item 59
            \item 100
            \item 1200
            \item 1092
            \item 1000000
        \end{enumerate}
    \end{multicols*}
    \item Define the Divide-by-2 method in pseudo-code.
    \item Convert the following decimal numbers to binary using \gls{divide-by-2}.
    \begin{multicols*}{2}
        \begin{enumerate}
            \item 7
            \item 2
            \item 89
            \item 47
            \item 242
            \item 574
            \item 1655
            \item 6757
        \end{enumerate}
    \end{multicols*}
    \item Convert the following binary numbers to decimal using the table method:\label{hw:tablemethod}
    \begin{enumerate}
        \item 0
        \item 1
        \item 1010
        \item 1111
        \item 1110101
        \item 1010100101
        \item 1000000000
        \item 1000100101
    \end{enumerate}
    \item \label{dividebyn} Convert the following numbers to base-5 using the Divide-by-5 method.
    \begin{multicols*}{2}
        \begin{enumerate}
            \item 8
            \item 6
            \item 651
            \item 470
            \item 7927
            \item 4674
            \item 18788
            \item 10328
        \end{enumerate}
    \end{multicols*}
    \item Repeat Problem \ref{dividebyn} the base of your favorite number greater than 20.
    \item Repeat Problem \ref{dividebyn} in base-128.
    \item Repeat Question \ref{hw:tablemethod} using the divide-by-2 method.
    \item Write a command-line program to convert a decimal number to an arbitrary number base.
    \item Write a command-line program to display the time in \gls{base-10} for the hour and minute. This implies that both the hours and number of minutes is out of 10 maximum rather 
    than a positional notation, and may require a fractional output. 
   \end{enumerate} 
\end{multicols*}
